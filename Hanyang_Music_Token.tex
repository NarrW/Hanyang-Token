\documentclass[conference]{IEEEtran}
\IEEEoverridecommandlockouts
% The preceding line is only needed to identify funding in the first footnote. If that is unneeded, please comment it out.
\usepackage{cite}
\usepackage{amsmath,amssymb,amsfonts}
\usepackage{algorithmic}
\usepackage{textcomp}
\usepackage{xcolor}
\def\BibTeX{{\rm B\kern-.05em{\sc i\kern-.025em b}\kern-.08em
    T\kern-.1667em\lower.7ex\hbox{E}\kern-.125emX}}
\begin{document}

\title{Hanyang Music Token(HMT)\\
{\footnotesize \textsuperscript{}}
\thanks{Identify applicable funding agency here. If none, delete this.}
}

\author{\IEEEauthorblockN{1\textsuperscript{st} Given Name Surname : Ganghyuk Ko}
\IEEEauthorblockA{\textit{dept. name of organization (of Aff.) : Information System} \\
\textit{name of organization (of Aff.) : Hanyang University}\\
City, Country : Seoul, Rep. Korea \\
email address : over2ture@gmail.com}
\and
\IEEEauthorblockN{2\textsuperscript{nd} Given Name Surname : Taewook Kang}
\IEEEauthorblockA{\textit{dept. name of organization (of Aff.) : Information System} \\
\textit{name of organization (of Aff.) : Hanyang University}\\
City, Country : Seoul, Rep. Korea \\
email address : taewook1000@naver.com}
\and
\IEEEauthorblockN{3\textsuperscript{rd} Given Name Surname : Gyusik Hahm}
\IEEEauthorblockA{\textit{dept. name of organization (of Aff.) : Information System} \\
\textit{name of organization (of Aff.) : Hanyang University}\\
City, Country : Seoul, Rep. Korea \\
email address : gyusik6314@naver.com}
\and
\IEEEauthorblockN{4\textsuperscript{th} Given Name Surname : IBN JOBAER MD SAKIB}
\IEEEauthorblockA{\textit{dept. name of organization (of Aff.) : Computer Science} \\
\textit{name of organization (of Aff.) : Hanyang University}\\
City, Country : Seoul, Rep. Korea \\
email address : sakib0955@gmail.com}
}

\maketitle

\begin{abstract}
As the fourth industrial revolution and the need for technological advancement are emerging, the block chain industry is getting popular recently. A variety of the latest technologies are based on block-chain networks, and the field of cryptography is of great interest. Cryptography, unlike the general currency issued by governments or central banks, creates values and ecosystems according to rules set by the original inventor. Cryptography is handled in a distributed system approach that utilizes block-chain technology. Those participating in the decentralized system are called miners, who receive a coin-type fee as compensation for block-chain processing. Since the crypto currency is maintained with this structure, the production cost due to the issuance of money is not paid at all, and the transaction cost such as the transfer cost can be greatly reduced. In addition, since it is stored in a computer hard disk or the like, it is advantageous in that it does not incur a storage cost and functions as a value storing means because there is no fear of stolen or lost. HMT is a system that helps the production and consumption of the music market and the activity of various artists including indie and pro. It combines cryptography technology and P2P file sharing network. Through HMT, artists transmit their work directly to the block-chain network through the smart contract. These contracts automatically distribute revenue and distribution, and artists have rights to royalties on their own. Listeners can easily retrieve and listen to music whenever they want, pay for artists directly through passwords while listening to music videos, and support artists. Decentralized, transparent platforms allow artists and listeners to connect directly.
\end{abstract}

\section{Introduction}
\subsection{Cryptocurrency concept}
Commerce on the Internet today has come to rely entirely on electronic payment methods with financial institutions as 3rd party credit institutions. The system works well for most transactions, but still has an inherent weakness as a credit-based model.
Completely revocable transactions are virtually impossible because financial institutions cannot avoid mediating transactional disputes. Such arbitration costs ultimately raise transaction fees, limit the actual minimum transaction value to prevent the possibility of micro-transactions, and even make payments that are irrecoverable even more costly. In other words, more credits are required to reverse the payment. Traders are asking for more information that is unnecessary, making them bothered and alert. It is an inevitable reality to make fake payments at a certain rate. These costs and uncertainty of payment can be avoided by people physically paying the money directly, but there is no way to pay for it without a credit institution. The problem is that by using an electronic payment system based on encryption technology rather than credit, two voluntary traders make direct transactions possible without a 3th party credit institution. Computationally irreversible remittances can protect sellers against fake payments, and buyers can be protected through a common escrow method. Crypto currency uses a P2P distributed network-based timestamp server to make it possible to prove the time sequence of transactions in a computational way. This system is secure as long as honest nodes collectively control more computing power than rogue cooperating node groups.

\subsection{ERC-20}
The ‘ERC’ in ‘ERC20’ is the abbreviation for "Ethereum Request for Comments”. Precisely speaking, ERC is the official protocol responsible for proposing improvements to the Ethereum Network. The number ‘20’ in ‘ERC20’ just signifies the unique proposal ID.
In the last couple of years, the “ERC20 Standard Protocol” has played a pivotal role of an enabler, around which the entire Blockchain, Cryptocurrency industry has flourished. The ERC20 protocol has brought in the much needed standardization—which was obviously missing prior to the inception of ERC20 protocol—which accelerated the development of hundreds and thousands of DApps (Distributed Applications) onto a universally standard platform.
The “ERC20 standard protocol” also allowed all the developers, enterprises to tokenize their projects and conduct crowed funding, in the form of ICOs. Long story short, how much ever Blockchain, Cryptocurrency industry has accomplishment till now, a lot of it has been possible due to the advent of a standard protocol like ERC20. If the market-cap of Ethereum, all the other ERC20 tokens are put together, the figure would easily topple Bitcoin from its numero-uno position for its market-cap.

\subsection{Vision}
HMT is a system that helps the production and consumption of the music market and the activity of various artists including indie and pro. It combines Crypto currency technology and P2P file sharing network. At HMT, artists will be directly involved in the Ethereum block chain network through the smart contract, and revenue and distribution will be automatically generated. As a result, the artists themselves have rights to royalties. Listeners can listen to music easily whenever they want, watch music videos, and pay for artists directly through passwords to support their activities. The token is decentralized and direct to connect artists and listeners through a transparent platform. In the modern music market, listeners can access a variety of music through a gigantic entertainment market platform such as iTunes, but artists are experiencing a lot of technological and economic difficulties, and most of the revenue is taking music offering enterprise.
It is negative side. With the development of block chain technology, these problems may be complemented and the relationship between artists and listeners may be reestablished. With Crypto currency, you can connect fans and artists with immediate and minimal fees. 

\subsection{Background}
Billboards in the United States are so large that it accounts for 35% of the world music market. Recently, K-Pop is ranked 8th in the world and occupies the music market. In addition to recordings and digital sound sources, it also provides additional revenue from video content such as YouTube and M-net, and is expanding its influence in various countries such as the US and Europe beyond Asia. K-Pop grows with the content industries such as movies and dramas, which are making Korean wave trends, and is spreading influence at a faster pace through the common language of music all over the world.
HMT was inspired by the Korean passion for music. It has been noted that financial assistance must be supported to maintain the creative activities of artists and to continue their activities through market entry. Therefore, HMT aims to create a platform that can directly connect fans and artists using block-chain technology.


\section{Requirements}

\subsection{Develop and publish an Etherium Network (ERC-20) based cryptocurrency}
Develops and publishes a cryptocurrency token (HMT) based on the Ether Network.

\subsection{Develop a sample music program / app that utilizes the developed coin}
Develop a sample version of the music-related PC client / application that utilizes the developed coin.

\section{Development Environment}

\subsection{Etherium Network (ERC-20) based cryptocurrency.}\label{AA}
-	 NodeJS
Node is similar in design to, and influenced by, systems like Ruby's Event Machine or Python's Twisted. Node takes the event model a bit further. It presents an event loop as a runtime construct instead of as a library. In other systems there is always a blocking call to start the event-loop. Typically behavior is defined through callbacks at the beginning of a script and at the end starts a server through a blocking call like EventMachine::run(). In Node there is no such start-the-event-loop call. Node simply enters the event loop after executing the input script. Node exits the event loop when there are no more callbacks to perform. This behavior is like browser JavaScript — the event loop is hidden from the user.

-	 Truffle
A world class development environment, testing framework and asset pipeline for blockchains using the Ethereum Virtual Machine (EVM), aiming to make life as a developer easier.

-	 Solidity + OpenZeppelin
OpenZeppelin is a library for secure smart contract development. It provides implementations of standards like ERC20 and ERC721 which you can deploy as-is or extend to suit your needs, as well as Solidity components to build custom contracts and more complex decentralized systems.

\subsection{Music application utilizes the developed coin}
- Unity + C Sharp (VS2018)
A cross-platform game engine developed by Unity Technologies and used to develop application, games for PC, consoles, mobile devices and websites.

\subsection{Task distribution}
Name: 	Ganghyuk Ko /
Role: 	Software developer /
Task Description and etc.: 
Responsible for software implementation.

Name: 	Taewook Kang /
Role: 	Software developer, Development manager /
Task Description and etc.:
Managing software development and designing systems

Name: 	Gyusik Hahm /
Role: 	Manager, Customer /
Task Description and etc.:
Study and analyze what needs to be supplemented by using software.

Name: 	IBN JOBAER MD SAKIB /
Role: 	User, Customer /
Task Description and etc.
Develop best-quality software by adjusting user’s aspect and developer’s aspect

\section{Specifications}

\subsection{Develop and publish an Etherium Network (ERC-20) based cryptocurrency}
- ERC20 Protocol
How the HTTP protocol enabled the internet to scale to the levels that it has today, on similar grounds, ERC20 is a protocol standard that is driving the growth forward for the Blockchain space.
It outlines a set of commands that a token—on the Ethereum Network—must adhere to and implement. ERC20 is not a technology, nor software, nor a piece of code, it basically is a specification standard for tokens. If a token implements the prescribed technical specification, it becomes an ERC20 token.
The ERC20 protocol standard contains basic functions that any useful token should implement, in order to enable its trades on the exchanges. These include inquiring the balance of tokens at a certain address, approve the transfer of token, transferring tokens and the total supply of tokens

-	 ERC20 Token
It’s a universal fact that every person has an innate behavior, his or her individuality or flavor. This variable flavor becomes completely apparent amongst developers as well, as different developers achieve the same end result but with differently scripted code. This leads to different developers creating different interfaces for the same token.
For example, one developer may use the function name “transfer” for transferring the token but other developer may use the function name “Send” for the exact same functionality. This type of fragmentation leads to lot of redundant duplications and an inconsistent experience for the end user.
Before ERC20 standards came into existence, all new tokens that came out as ICOs on the Ethereum Network, implemented their own flavor of all the functions. Each one had its own list of functions for transferring tokens, function names and different arguments.
This kind of fragmentation, irregularity in coding standards lead to a number of challenges.
(1)	The standard defines a set of functions;
(2)	Name of the functions;
(3)	Arguments they take;
(4)	The return value they have;
(5)	 And the behaviors that is expected from these functions;
(6)	A set of events that need to be emitted from the token.
Each and every ERC20 compliant tokens needs to have the below six functions and two events as part of its interface definition:

In order for the token on Ethereum to be ERC20 compliant, it must implement the six functions and the two events. In addition to the standard function and events, one can have their own custom functions and events, primarily for token administrative activities.

-	 Advantages of Standardizing the Token Specifications
(1)	Uniformity Across the Tokens: It lead to the creation of standard tools for interacting with multiple types of token. For e.g. the token developers do not have to create tools of their own. Moreover they also don’t have to create their own specifications. Once an ERC20 compliant token gets created, anyone having knowledge of ERC20 standards can easily understand the behavior of the token.
(2)	Listing for Trading on Exchanges: A standard compliant token can easily be listed on any exchange that supports that standard such as ERC20 and trading on the exchange became a smooth process. The ease of use and ease of listing on exchanges has led to the creation of more and more ICOs on the Ethereum Network. This in turn has given birth to countless number of innovative projects in the Blockchain space.

\subsection{Music application utilizes the developed coin}
- Unity GUI
The application is developed based on Unity graphical user interface. AudioSource, VideoPlayer and Component are used when processing multimedia.

-	 Nethereum Library
Nethereum is a library that makes it easy to write JSON RPC in C Sharp provided by Ethrium client such as Geth or Parity in C Sharp.  Throught this library, it is possible to perform Ethernet transactions in various applications made with Unity. Etherium’s Smart Contract feature is also available. 
Nethereum, which is distributed for Unity, can use the following functions.

(1)	Web3 API
Below the Nethereum.JsonRpc.UnityClient namespace, there is a one-to-one correspondence functions with the web3 APIs. They can do Account balance inquiry, Transaction, Contract call and so on.

(2)	Private key create and sign
You can create ethereum accounts directly or use EthECkey functions in the Nethereum. Nethereum.Signer namespace to sign any data.

(3)	Offline sign
With TransactionSignedUnityRequest, you can sign inside the game when you are trading. With this feature, you can manage the private key of the user directly in the application without sending it to the ethereum node.

- Searching the music artists
You can browse the information of various music artists from all over the world registered in the application, listen to music or watch music videos. You can see and access individual fan sites and communities at a glance.

-	 Token payment and compensation to artist
Music artists issue their own music tokens, each of which has a hard-forked Hanyang Token. Fans can purchase individual music tokens for artists by paying Hanyang Token. The individual music tokens purchased are basically supporters. Thereafter, the tokens can perform the money that can be paid for performances, related products, and music releases. 

\end{document}
